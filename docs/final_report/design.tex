\documentclass{article}

\usepackage[latin1]{inputenc}
\usepackage[margin=1.0in]{geometry}
\usepackage{amsmath}
\usepackage{amsfonts}
\usepackage{amssymb}
\usepackage{array}
\usepackage{parskip}
\usepackage{graphicx}
%\usepackage{dsfont}
\usepackage{framed}
\usepackage{ upgreek }
\usepackage{listings}
\lstset{frame=single}
\lstset{basicstyle=\small}
\usepackage{fancyhdr}
\setlength{\headheight}{15.2pt}
\pagestyle{fancy}
\setlength{\parindent}{0pt}
\setcounter{secnumdepth}{2}
\usepackage{graphicx}
\usepackage{float}

%\usepackage{sectsty}

\begin{document}

% title stuff
\title{Interactive Fractal Viewer \\
CSEE W4840 Final Report}
\author{Nathan Hwang - nyh2105@columbia.edu \and
Richard Nwaobasi - rcn2105@columbia.edu \and
Luis Pe\~{n}a - lep2141@columbia.edu \and
Stephen Pratt - sdp2128@columbia.edu }

\maketitle
\newpage
\abstract{Fractals are often appreciated for their rich and elegant
  internal complexity. It is this complexity that is responsible for the
  beautiful aesthetic of these famed mathematical images as well as
  the amount of computational power required to generate them. Using
  fixed point calculations within parallelized sequential logic
  blocks, we aim to develop an hardware-accelerated fractal generator,
  capable of computing and displaying quadratic Julia sets in
  significantly less time than a software-based solution.}

% ------------------------------------------------------------
% and we begin the design document proper


% ------------------------------------------------------------
\section{Background}
\subsection{An Introduction to Julia sets}

The Julia set $J$ of a complex rational function $f: \mathbb{C} \rightarrow \mathbb{C}$ is can be expressed as:

$$
\forall z: \lim_{n \rightarrow \infty} |f^n(z)| = \infty
$$

That is, given a function that describes a mapping from the complex numbers to the complex numbers, that function's
Julia set is the set of numbers for which repeated application of that function results in convergence towards 
infinity. Julia sets have many remarkable properties. Iterations of a function is chaotic on its Julia set, meaning that it represents a dynamical system in which tiny perturbations in initial conditions can result in disproportionately large changes in 
the evolution of the system. As chaotic systems, they can be used to generate pseudo-random numbers. When plotted,
Julia sets have the capacity to produce stunning images.

Quadratic polynomial Julia sets are Julia sets of functions that take the form

$$
f_c(z) = z^2 + c
$$

So any given quadratic polynomial Julia set is uniquely described by some point on the complex plane $c$. These sets produce fractals, and so they often exhibit self-similarity. Quadratic polynomial Julia sets have the nice property that the magnitude of $f_c(z)$ at any point in a recurrence whose initial value was \textbf{not} in the Julia set will always be bounded above by $2$. Thus, any point $z$ that causes $|f_c(z_i)| > 2$ for any $z_i$ in the recurrence is necessarily \textbf{not} a member of $J(f_c)$

\subsubsection{Generating Quadratic Polynomial Julia Sets}

Because points not belonging to the Julia set of some quadratic polynomial $f_c$ (also called points in the Fatou set of $f_c$) are guaranteed to be bounded in magnitude by $2$ during repeated application of $f_c$, one may generate the Julia set for a given window in $f_c$ by sampling the members of that window, repeatedly applying $f_c$, and testing whether or not the magnitude ever breaks away from $2$, giving up after some fixed number of iterations. This is the basic design of the \textbf{Interactive Fractal Viewer}. 

When plotting the resulting Julia set, one may choose to colorize the image based on how long it took for the iteration beginning at each point in the complex plane to become unbounded. We shall henceforth call this value the breakaway iteration, or $k$. Thus, the problem of plotting a Julia set on a screen reduces to computing the viewing window as a section of the complex plane, and finding $k$ for each sample point in that section.

Generating Julia sets in this fashion can be a very lengthy process, especially when the points must be done in series. However, since each recurrence to be tested is completely independent of each other, the problem is extraordinarily parallelizeable. Furthermore, since the function being applied is relatively simple, it may be very easily implemented in hardware. Giving the \textbf{Interactive Fractal Viewer} its motivation.

\begin{figure}
  \centering
	\includegraphics[width=200pt]{images/julia_wiki.png}
  \caption{The quadratic polynomial Julia set defined by $c=(\phi - 2)+(\phi - 1)i$ (from Wikipedia)}
\end{figure}

\section{Design}

\begin{figure}\label{fig:block}
  \centering
	\includegraphics[width=\textwidth]{block_diagrams/top_level.pdf}
  \caption{High-level Block Diagram}
\end{figure}

\subsection{High-level Overview}

The following is a description of how data travels through the block structure specified in Figure~\ref{fig:block} when generating a single fractal.

\begin{itemize} 

\item Data flow originates from the \textbf{NIOS II Processor}, which initializes the system
by computing a set of parameters that can be used to describe the target window and 
fractal. The \textbf{NIOS} writes these parameters across the \textbf{Avalon Bus} onto an on-board \textbf{RAM} 
and sends instructions to generate a new image to a special instruction register.

\item When the instruction register receives the generate signal, it tells a component
charmingly referred to as the \texttt{Rammer} to write the parameters information contained 
in the \textbf{RAM} into registers. It then asserts a \textit{generate} signal that will be read by 
the \texttt{Window Generator}


\item The \texttt{Window Generator} takes these parameters as input and builds a set of 4-tuples $(x, y, a, b)$ where
  each tuple is a mapping from a \textbf{VGA} coordinate $(x, y)$ to a value in
  the complex plane of the form $a+bi$. The $(a, b)$ values eventually be used to compute a breakaway count $k$ for
  each co-ordinate on the screen. The computation will be performed by a specialized component called an 
  \texttt{Iterative Function Module}, or \texttt{IFM}.

\item Before the next $(a, b, x, y)$ tuple is delivered to an \texttt{IFM}, it must be requested by a component
known as the \texttt{IFM Controller}. The \texttt{IFM Controller} is responsible for distributing work among
several \texttt{IFM}s working in parallel. 

\item When an \texttt{IFM} enters a ready state, it is given the next $(a, b, x, y)$ tuple and will begin performing
the function iterations using the constant prescribed by the set to be generated. The \texttt{IFM} transitions into a 
done state after a fixed number of iterations, or when the squared magnitude of its current iteration exceeds $4$.

\item The \texttt{IFM Controller} takes the data given by the done-state \texttt{IFMs} and writes it to a queue
that will deliver it to the \texttt{Coordinate-Breakaway Lookup Table}, implemented using the on-board \textbf{SRAM}. 
The \texttt{IFM} data takes the form of the $(x, y, k)$ triples we set out to create. The $k$ value of each triple
is stored in the \texttt{Coordinate-Breakaway Lookup Table} at an address determined by the $(x, y)$ values. In this 
way, we map \textbf{VGA} coordinates to their associated breakaway iterations.

\item The \texttt{VGA Module} fetches results from the \texttt{Coordinate-Breakaway
  Lookup Table} and colorizes them using a separate \textbf{ROM}-based \texttt{Colorization Lookup Table}.
  The \texttt{Colorization Lookup Table} takes a breakaway count, $k$, and maps it to an $(r, g, b)$ bit-vector for use by the \textbf{VGA}.
\end{itemize}


% ------------------------------------------------------------
% now, we delve into the lower levels of the implementation
% ------------------------------------------------------------
\subsection{Critical Modules}

% ----------------------------------------
\subsubsection{User Interface Module}

As an interactive device, our fractal generator has the capacity to accept user parameters such as window size or
Julia set constants during operation. This communication with the user is facilitated by the UI module, which is 
implemented on the NIOS II processor. This module is responsible for handling communication with input peripherals
and translating user input into information that can easily be used by the hardware-based fractal generator. Once 
this information has been translated into a set of instructions for the generator, these instructions are written
to a specialized bus across the Avalon interconnect fabric.

We have the Nios II configured to use the SDRAM as its memory store, making the SRAM available for other uses.

\begin{figure}
  \centering
    \includegraphics[width=200pt]{block_diagrams/nbat.pdf}
  \caption{Block and Timing Diagram of the Avalon interconnect fabric. Provided by Altera Corporation.}
\end{figure}

Parameters of primary concern are those of viewing window and Julia set constant. The window is set using the 
following values (which will be elaborated on in the next section)

\begin{itemize}
\item $a_{min} - 36$ bits
\item $a_{diff} - 36$ bits
\item $a_{leap} - 10$ bits
\item $b_{min} - 36$ bits
\item $b_{diff} - 36$ bits
\item $b_{leap} - 10$ bits
\end{itemize}

Meanwhile, the Julia set constant is set using the following values

\begin{itemize}
\item $c_{real} - 36$ bits
\item $c_{img} - 36$ bits
\end{itemize}

\textit{New RAM stuff goes here}

\subsubsection{Window Generator}

The \texttt{Window Generator} serves to kick off the calculation cascade,
computing the position in the complex plane represented by each pixel in the
given input window, thereby producing 
$(x, y, a, b)$ tuples. 

\begin{figure}
  \centering
    \includegraphics[width=160pt]{block_diagrams/win_gen.pdf}
  \caption{High-level Block Diagram of the Window Generator}
\end{figure}


The generator uses a specialized procedure that requires only addition and comparison
operations to map out a whole window. Say we have a window that stretches from $v_{min}$ to $v_{max}$ over
N pixels. The procedure works by iterating from 0 to N-1 and producing a sum at each step of the way that
corresponds to a the value of $v$ at that point. The procedure requires a few values as input:


\begin{itemize}
\item $a_{min} - 36$ bits: Describes the minimum value of $a$ in the window
\item $a_{diff} - 36$ bits: Describes the standard differential between consecutive a values in the window $(a_{max} - a_{min})/WIDTH_{SCREEN}$ this is computed by the NIOS processor.
\item $a_{leap} - 10$ bits: Periodically, we will need to add 1 to our sum to compensate for precision loss. This value
corresponds to the length of the intervals between these "leap cycles" $WIDTH_{SCREEN}/((a_{max} - a_{min})\%WIDTH_{SCREEN})$
\item $b_{min} - 36$ bits: Describes the minimum value of $a$ in the window
\item $b_{diff} - 36$ bits: Describes the standard differential between consecutive a values in the window $(b_{max} - b_{min})/HEIGHT_{SCREEN}$ this is computed by the NIOS processor.
\item $b_{leap} - 10$ bits: Periodically, we will need to add 1 to our sum to compensate for precision loss. This value
corresponds to the length of the intervals between these "leap cycles" $HEIGHT_{SCREEN}/((b_{max} - b_{min})\%HEIGHT_{SCREEN})$
\end{itemize}

The window generator is therefore comprised of two "differential counters" that are responsible for performing the 
iterations. One computes values for $b$ and the other $a$. 

\begin{figure}
  \centering
    \includegraphics[width=160pt]{block_diagrams/acounter.pdf}
  \caption{Block diagram of a differential counter used in window generation}
\end{figure}

When the differential counter receives a reset signal, it initializes its data according to the signals coming in.
Then, each time it recieves a next-value signal, it increments the output value accordingly. If the counter reaches
its maximum, it asserts a flag.

\begin{figure}
  \centering
    \includegraphics[width=200pt]{state_diagrams/diff_counter.pdf}
  \caption{State Diagram for the diff counter, Moore machine: signals
    \texttt{max=c\_cuis=max\_itr}, \texttt{leap=iter\_count=v\_leap}. Omit
    unused signals (X) for compactness. Colorcoded signal bundles.}
\end{figure}

\begin{figure}
  \centering
    \includegraphics[width=200pt]{timing_diagrams/diff_counter.pdf}
  \caption{Timing diagram of the differential counter}
\end{figure}

In the window generator, the differential counters are hooked up in such a way that the points are cycled through
from left to right down the screen.

\begin{figure}
  \centering
    \includegraphics[width=150pt, angle=270]{block_diagrams/win_gen_interior.pdf}
  \caption{Block diagram illustrating $(x, y, a, b)$ tuple dataflow.}
\end{figure}

The window generator takes reset signals from the bus connected to the NIOS processor. When the window
generator is ready for computation (the same cycle that it is reset by the bus), it asserts a data flag.
When the IFM reads an $(x, y, a, b)$ tuple, it asserts a next value signal indicating that it will need new data in the following cycle. Once the window generator runs out of values to give, it asserts an at max flag.

\begin{figure}
  \centering
    \includegraphics[width=180pt]{state_diagrams/diff_window_gen.pdf}
  \caption{State Diagram for the window generator code, Moore machine:
    signals \texttt{max=c\_cuis=max\_itr},
    \texttt{leap=iter\_count=v\_leap}. Omit unused signals (X) for
    compactness.}
\end{figure}


\begin{figure}
  \centering
    \includegraphics[width=160pt]{timing_diagrams/gen_ifm.pdf}
  \caption{Timing diagram of interface between the window generator
    and the IFMs.}
\end{figure}

\subsubsection{Parallel IFM Control Module}

Rendering Julia set fractals requires many iterations of relatively
simple computations in the complex plane. This sequence of
computations is independent for each point in the image, which is why
the calculation of fractal sets lends itself to parallel
computation. However, the very nature of the iterated fractal
calculation means that the amount of time spent performing computations on each individual point
can vary drastically, introducing synchronization issues. It is the responsibility of the
IFM control module to resolve these issues.

\begin{figure}
  \centering
    \includegraphics[width=160pt]{block_diagrams/ifmunit.pdf}
  \caption{Block diagram for the IFM wrappers}
\end{figure}


The IFM control module constantly transmits the $(x, y, a, b)$ tuple currently being expressed by the window generator to 
each of the IFMs. When an IFM indicates that it is in the ready state, the controller asserts a signal instructing the 
IFM to accept the new data and begin computation (assuming that the window generator is asserting the valid data flag).
Simultaneously, the controller signals to the window generator that it needs the next data tuple in the window. If more
than one IFM is in the ready state at once, the controller only sends the read and compute signal to one, saving the 
upcoming data tuples for the rest.


% ----------------------------------------
\subsubsection{Iterative Function Module (IFM)}

A quadratic polynomial Julia set is generated by applying the function

\begin{equation}
f_c(z) = z^2 + c
\end{equation}

repeatedly, where $z,c \in \mathbb{C}$. For any given pair $(z, c)$, this recurrence will result in one of two outcomes:
\begin{itemize}
\item The magnitude of the complex values generated by the recurrence may stay bounded by $2$
\item The magnitude may become unbounded and escape toward infinity
\end{itemize}

A point $z$ on the complex plane is in the Julia set uniquely defined by the complex number $c$ if and only if the 
recurrence remains bounded for $(z, c)$. To determine whether or not a point remains bounded for a given $c$, we compute
a fixed number of iterations on the recurrence (in our case $127$) and report the iteration in which the value generated has
a squared magnitude of greater than $4$. Those points that do not become unbounded in this many iterations are considered
to be part of the set.

Because the factors of the multiplication are complex numbers, computing their product involves
$3$ real-number multiplications. For $z = a + bi$ we compute\\

\begin{eqnarray}
P_A &=& a^2 \nonumber \\
P_B &=& b^2 \nonumber \\
P_C &=& ab \nonumber
\end{eqnarray}

With these values we can compute:

\begin{eqnarray}
a_{next} &=& P_A - P_B + c_{real}\nonumber \\
b_{next} &=& 2P_C - c_{img} \nonumber \\
|z|^2 &=& P_A + P_B \nonumber 
\end{eqnarray}

The squaring operations for $P_A$ and $P_B$ is perfomed by a specialized logical circuit provided as an \textit{Altera Megafunction}. The multiplication $P_C$ is perfomed by embedded multipliers on the DE2.

\begin{figure}[H]
  \centering
    \includegraphics[width=160pt]{block_diagrams/ifm.pdf}
  \caption{Arithmetic Logic Circuit within each \texttt{IFM}}
\end{figure}

Real-valued numbers are represented as two's-complement fixed-point binary
values in our circuit. A complex number is comprised of two such data vectors. 
We restrict ourselves to 36 bits, as the onboard multipliers
are sized as such. 

In order to accomodate the largest-magnitude value we
expect to come across during any iteration, we require 6 bits to the
left of the radix. Thus, our fixed-point values have 30 bits to the
right of the radix.

\begin{figure}[H]
  \centering
    \includegraphics[width=200pt]{state_diagrams/ifm.pdf}
  \caption{State diagram for a single \texttt{IFM}. Moore
    machine: using abstract transition descriptions. Omits unused
    signals for compactness.}
\end{figure}


To more easily facilitate communication with the \texttt{IFM Controller}, each \texttt{IFM} is contained within a wrapper module.
Thus, the \texttt{IFM Controller} need only alter the state of the wrapper module, and the wrapper module will transmit signals to the \texttt{IFM}s indicating the desired behavior.

\begin{figure}[H]
  \centering
    \includegraphics[width=200pt]{state_diagrams/ifmunit.pdf}
  \caption{State diagram for a single \texttt{IFM} unit (\texttt{IFM} handler/wrapper),
    Moore machine: using abstract transition descriptions.}
\end{figure}

If a wrapper module is in the done state, the controller indicates that its $(x, y, c)$ triple should be read into
the output register. If multiple IFM wrappers are in the done state simultaneously, the controller chooses one at a
time to be read in. These triples are then augmented with an asserted write enable flag to indicate that they represent
valid data, and should be written to the Coordinate-Breakaway lookup table. 


% ----------------------------------------
\subsubsection{Coordinate-Breakaway Lookup Table}

After the count associated with each pixel is calculated, it must be
stored in a framebuffer that interfaces both with the Parallel IFM
Control Module as well as the VGA module.

\begin{figure}[H]
  \centering
    \includegraphics[width=160pt]{block_diagrams/clut.pdf}
  \caption{Block Diagram of the Coordinate-Breakaway LUT: signals on
    the right side are inputs, signals on the left are outputs.}
\end{figure}

For this, we use the SRAM chip that is built into the DE2 board, for
its relatively expansive memory size (versus on-chip memory), fast
speed, and ease of use (versus the SDRAM chip). The SRAM chip has a 
512kibibytes capacity that can be accessed and written to in half a 50MHz
clock cycle, making it ideal for our purposes.

Since we display a $640\times 480$ image in the VGA module and keep 8
bits of iteration information for each pixel, we need a grand total of
300kibibytes to store the information, fitting well within the
confines of the given 512kibibyte SRAM chip.

We use a straightforward addressing scheme to store the count
information, using the $y$ position as the top 9 bits of the address,
and the $x$ position as the bottom 10 bits of the address. This way,
finding the address from a given pixel position is very fast.

A small wrinkle is the fact the SRAM is in fact a $256K\times 16$ bit memory,
reading and writing in 16 bit chunks. This merely means that the very
bottom bit of the $x$ position does not go to the address, but is
routed to the bitmask signal indicating whether the byte sought is in the upper or lower half.
Of the 16-bit word that is addressed by the remaining 18 bits.

\paragraph{Reading/Writing}

Since the SRAM has only one IO port, reads and writes must be time
multiplexed. The VGA module will be consistently requesting data from the SRAM at 25MHz. However, while 
the fractal is being generated, the IFMs will be providing information that must be written to the SRAM 
at the same frequency. This means that we must interleave reads and writes to the SRAM.

We can use the structure of the reads from the VGA to our
advantage to make room for the necessary writes. Reads always follow a pattern, where
if we read the lower half of a 16bit word, then we will read the
higher half in the next 25MHz clock cycle. Hence, when we require the lower
half of a word, we can fetch the entire word in one read, save the
higher half in a register, and return it when it's required in
the next clock cycle. In this way, we reduce the frequency of VGA reads from the SRAM to every other cycle on a 
25MHz clock.

\begin{figure}[H]
  \centering
    \includegraphics[width=200pt]{timing_diagrams/clut_vga.pdf}
  \caption{Timing diagram of the interface between the Coordinate-Breakaway LUT and VGA module}
\end{figure}

Even with every other 25Mhz cycle being dedicated to writing the data being sent out by the IFMs, the SRAM might
still miss a coordinate if the IFMs are generating their maximum possible throughput of 25MB/s. To account for this, 
we put the junction serving writedata to the SRAM on a 50MHz clock. This junction consists of a shift register that 
constantly reads from the IFM output, but only shifts when the SRAM's read enable signal is not being asserted. 

\begin{figure}[H]
  \centering
    \includegraphics[width=200pt]{state_diagrams/clut.pdf}
  \caption{Simplified State Diagram of the Coordinate-Breakaway LUT:
    Mealy machine, only includes re and we as inputs and rv as an
    output, with $\thicksim$ denoting a lack of outputs}
\end{figure}

This means that a given tuple $(x_j, y_j, c_j, we_j)$ being output by the IFM controller will be overwritten by the 
following $(x_k, y_k, c_k, we_k)$ output during every VGA read cycle. However, since the shift register is reading from a 25MHz process at 50MHz, it is guaranteed that every write tuple will be read into the shift register twice. Because the IFM
and VGA controllers are synchronized, the VGA read always occurs when $j \neq k$. But because every tuple is read into the
shift register twice, the value that was overwritten, $(x_j, y_j, c_j, we_j)$, must also exist in the next cell over, 
guaranteeing reliable data transmission.

\begin{figure}[H]
  \centering
    \includegraphics[width=200pt]{timing_diagrams/ifm_clut.pdf}
  \caption{Timing diagram of the interface between the IFMs and the Coordinate-Breakaway LUT.}
\end{figure}

Dangerous though it may be, the effect of having a faster clock for the SRAM write junction is well contained.
No processes depend on the state of the write junction, and the junction is free to produce redundant write data
without consequence. The write junction merely serves as a conduit through which writedata is transmitted.


\subsubsection{VGA Module}

In order to display the generated Julia set, we connect a VGA controller 
 to the Coordinate-Breakaway lookup table. As the controller cycles
through output coordinates within the display area, it modifies the
read address signal for the lookup table. The data signal coming from
the RAM is thus the breakaway value associated with that coordinate.

\begin{figure}[H]
  \centering
    \includegraphics[width=200pt]{block_diagrams/vga.pdf}
  \caption{Block diagram of the VGA module}
\end{figure}

This breakaway value is passed through a decoder known as the
Colorization Lookup Table and the resulting $(R, G, B)$ signal tuple
is sent to the VGA port.

% ----------------------------------------
% figures for the VGA


\begin{figure}
  \centering
    \includegraphics[width=200px]{state_diagrams/vga.pdf}
  \caption{State diagram for the \texttt{VGA} module, Mealy machine: $\thicksim$
    stands in for no input/output, and VGA stands in for all the VGA\_
    signals (\texttt{VGA\_CLK}, \texttt{VGA\_HS}, \texttt{VGA\_VS},
    \texttt{VGA\_BLANK}, \texttt{VGA\_SYNC}, \texttt{VGA\_R}, \texttt{VGA\_G},
    \texttt{VGA\_B}). Omits unused signals for compactness.}
\end{figure}


% ------------------------------------------------------------
\subsection{Parametrization Modules}

\subsubsection{Software-based Parameterization}

\subsubsection{Colorization}

%These modules might be used parameterize the output fractal

%\begin{itemize}
%\item PS/2 Keyboard input can be used to allow the user to specify
%  fractal recomputation using a different set of parameters,
%  permitting the modification of window ranges and Julia set
%  constants. Keyboard input would be facilitated through the Nios II
%  processor and would require a reexecution of the Window Generator.
%\item We could create a module for permuting the display colors given
%  by the Colorization Lookup table using a periodic function, thus
%  causing the colors to cycle. This module would allow us to modify
%  the way the fractal looks without recomputing it.
%\end{itemize}

\section{System Performance and Validation}

\section{Milestone Report}

\begin{tabular}{rlp{4.5cm}p{4.5cm}}
\textbf{Milestone}&\textbf{Date}&\textbf{Goal}&\textbf{Accomplished}\\ \hline
&&&\\
Milestone 1&Mar 27&Have a static Julia set filled into a buffer.&
	Develop a \texttt{Window Generator} capable of communicating with a parallelized \texttt{IFM Controller}\\
&&&\\
Milestone 2&Apr 10&Display the colorized Julia set through \textbf{VGA}.&\textit{As planned.}\\
&&&\\
Milestone 3&Apr 24&Implement parameter mutation, with subsequent updates to the displayed Julia set.&
	Hardware-based parameter mutation and set redraw.\\
&&&\\
Deadline&May 10&Feature complete system with accompanying report and presentation.&\textit{As planned.}\\
&&&\\
\end{tabular}

\section{Challenges and Lessons Learned}

\section{Reflections and Prospective}

All things considered, our team is extraordinarily satisfied with the results that we've achieved. 
Development was certainly not without its ups and downs, but ultimately, we brought all major project
goals to fruition. 

Of course, like all good engineers, we can never claim to be completely satisfied with the fruits of our
labor. The following are a few features that we would like to implement once we have more time 
to work with the project:

\begin{enumerate}
\item Though window and constant selection are currently software-mutable system parameters, the UI through
which these parameters are modified does not allow very intuitive control of their values. A primary reason
for this is that our original system architecture was built with the explicit purpose of drawing fractals on the
screen, not anything else. In the future, we would like to let users select a custom window by modifying a frame 
on-screen. Explicit, rather than relative specification of new seed constants for the Julia set would also be 
desirable.
\item There exist a few timing glitches that were never completely resolved. Ironing out these glitches is a high
priority for future work.
\item It is our professional opinion that the color-cycle mode is totally radical. Having this cycling rate vary 
with audio input would turn our Interactive Fractal Viewer into a spectacular music visualizer.
\end{enumerate}

\appendix
\section{Source Code}
\subsection{VHDL}
\subsubsection{ifv.vhd}							%All
\subsubsection{ramcon.vhd}							%Luis
\subsubsection{rammer.vhd}							%Luis
\subsubsection{window\_gen.vhd}					%Steve	
\subsubsection{diff\_counter.vhd}					%Steve
\subsubsection{hook.vhd}							%Luis
\subsubsection{ifmd.vhd}							%Luis
\subsubsection{ifmunitd.vhd}						%Luis
\subsubsection{sram.vhd}							%Nathan
\subsubsection{vga.vhd}							%Richard / Edwards
\subsubsection{vgamod.vhd}							%Richard
\subsubsection{test\_filename.vhd}					%Steve
\subsubsection{Other VHDL Sources / Libraries}

\subsection{C}
\subsubsection{ifv.c}								%Steve / Richard
\subsubsection{ps2\_keyboard.h}					%Altera
\subsubsection{ps2\_keyboard.c}					%Altera
\subsubsection{ps2\_up\_ps2\_port.h}				%Altera
\subsubsection{ps2\_up\_ps2\_port\_regs.h}			%Altera
\subsubsection{Other C Sources / Libraries}

\subsection{Python}
\subsubsection{julia\_gen.py}						%Steve
\subsubsection{test\_ifm.py}						%Steve
\subsubsection{test\_ifm\_utils.py}				%Steve
\subsubsection{test\_image.py} 					%Nathan
\subsubsection{color\_gen.py} 						%Nathan
\subsubsection{Other Python Sources / Libraries}

\end{document}

