\documentclass{article}

\usepackage{sectsty}
\usepackage{amssymb}
\usepackage{amsmath}
\usepackage{graphicx}
\usepackage{listings}

\begin{document}

% title stuff
\title{Interactive Fractal Viewer \\
CSEE W4840 Project Design}
\author{Nathan Hwang - nyh2105@columbia.edu \and
Richard Nwaobasi - rcn2105@columbia.edu \and
Stephen Pratt - sdp2128@columbia.edu \and
Luis Pe\~{n}a - lep2141@columbia.edu}
\maketitle
\newpage
\abstract{Fractals are often appreciated for their rich and elegant
  internal complexity. This complexity is responsible for the
  beautiful aesthetic of these famed mathematical images as well as
  the amount of computational power required to generate them. Using
  fixed point calculations within parallelized sequential logic
  blocks, we aim to develop an hardware-accelerated fractal generator,
  capable of computing and displaying quadratic Julia sets in
  significantly less time than a software-based solution.}

% ------------------------------------------------------------
% and we begin the design document proper



% ------------------------------------------------------------
\section{High-level Overview}

The following is a description the high-level block structure of the project:

\begin{itemize}
\item A Window Generator builds a set of 4-tuples $(x, y, a, b)$ where
  each tuple is a mapping from a VGA coordinate $(x, y)$ to a value in
  the complex plane of the form $a+bi$. The Window Generator is
  implemtented by the Nios II processor.
\item The processor writes its values across its bus into a queue,
  which feeds values to Iterative Function Modules (IFMs).
\item Each IFM computes the number of function iterations required for
  a specified value to become unbounded. Multiple IFMs work in
  parallel.
\item Another queue recieves 3-tuples $(x, y, c)$ from the IFMs, where
  $(x, y)$ corresponds to a VGA coordinate and $c$ is the breakaway
  constant computed by the IFM. These values are passed one at a time
  to a buffer known as the Coordinate-Breakaway Lookup Table.
\item The VGA module fetches results from the Coordinate-Breakaway
  Lookup Table and colorizes them using a separate RAM-based lookup
  table, displaying the result.
\end{itemize}

These encompass only the so called ``critical modules'', what is
absolutely needed to have a fast draw of the fractal. Further work,
known as parameterization modules, offer various ways to mutate the
parameters used to draw the fractal.

\newpage

\begin{figure}[h!]
  \centering
	\includegraphics[width=\textwidth]{block_diagrams/top_level.pdf}
  \caption{High-level Block Diagram}
\end{figure}
\newpage

% ------------------------------------------------------------
% now, we delve into the lower levels of the implementation
% ------------------------------------------------------------
\section{Critical Modules}

% ----------------------------------------
\subsection{Window Generator}

The window generator serves to kick off the calculation cascade,
calculating the position of each pixel in the complex plane and
writing both to active memory linked up to the input queue (described
below).

In more detail, the module loops through each pixel given the screen
resolution, and maps it according to the translation of the origin as
well as the scale of the complex plane with respect to the VGA
screen. Since we may yet implement the translation of the origin as a
parameter mutable by the user, this seperation of concerns makes
sense.

We switched this module from a hardware implementation to code sitting
on top of the \verb!Nios II! processor. The processor uses a
specialized procedure to generate the window tuples. The procedure
requires only four divisions and two modulo operations, which are
executed once at procedure start time. The remainder of the operations
are additions and comparisons.

\begin{lstlisting}[caption="Window Generation Procedure"]
int row, col;
   
//configure our window
int y_max = /*max y value in window*/
int x_max = /*max x value in window*/
int y_min = /*min y value in window */
int x_min = /*min x value in window */

// set our iteration params

// to compute a and b, we will iterate through each
// pixel of the screen adding win_dim/screen_dim to a
// sum that starts at win_dim_min

// we will also have "leap" iterations, which are
// iterations in which the sum must be incremented by
// 1 to keep it on track towards screen_dim

// amount to add each iteration
int x_delt = (x_max - x_min)/VGA_WIDTH;
int y_delt = (y_max - y_min)/VGA_HEIGHT;

// leap total is the number of times we'll need to
// increment our sum by 1
int x_leap_total = (x_max - x_min)%VGA_WIDTH;
int y_leap_total = (y_max - y_min)%VGA_HEIGHT;

// leap count will keep track of how many times we
// incremented our count by this value
int x_leap_count = 0;
int y_leap_count = 0;

// leap interval is the number of cycles between
// leaps
int x_leap_interval = VGA_WIDTH/x_leap_total;
int y_leap_interval = VGA_HEIGHT/y_leap_total;

// time since last leap
int x_last_leap = 0;
int y_last_leap = 0;

long a = x_min;
long b = y_min;

// initialize a data value that will be written to
// the board

printf("here...\n");
// bottom to top so our min is in the bottom left
// corner
for(row = VGA_HEIGHT-1; row >= 0; row--)
{

a = x_min;

// left to right across the row
for(col = 0; col < VGA_WIDTH; col++)
{
    //concatenate bit vector
    //printf("a:%x\n",a);
    //printf("b:%x\n",b);
    //printf("here %d\n", foo++);
    if(a < 0)
        a |= (1 << AB_SIGN_BIT);
    if(b < 0)
        b |= (1 << AB_SIGN_BIT);
                   
    //write to board
    //IOWR_32DIRECT(JULIA_CALC_BASE, 0, (0x3));
    //printf("writing...\n");
    IOWR_32DIRECT(JULIA_CALC_BASE, 0, ((a<<4)|0x0));
    IOWR_32DIRECT(JULIA_CALC_BASE, 0, ((b<<4)|0x1));
    IOWR_32DIRECT(JULIA_CALC_BASE, 0,
                  ((((col<<X_SHIFT)|row)<<4)|0x2));
    IOWR_32DIRECT(JULIA_CALC_BASE, 0, (0x3));
   
    //printf("written\n");
    //printf("done\n");           
   
    //increment the imaginary value
    a+= x_delt;           
   
    //leap if necessary
    if(x_last_leap >= x_leap_interval &&
       x_leap_count < x_leap_total)
    {
        a++;
        x_last_leap = 0;
        x_leap_count++;
    }
    else
        x_last_leap++;          
   
}

//increment the imaginary value
b += y_delt;

//leap if necessary
if(y_last_leap >= y_leap_interval &&
   y_leap_count < y_leap_total)
{
    b++;
    y_last_leap = 0;
    y_leap_count++;
}
else
    y_last_leap++;

}
\end{lstlisting}

We have the Nios II configured to use the SDRAM as it's memory store,
since we use the, making the SRAM available for other uses.

% ----------------------------------------
% figures for the window generator

\begin{figure}[h!]
  \centering
    \includegraphics[width=\textwidth]{block_diagrams/win_gen.pdf}
  \caption{High-level Block Diagram of the Window Generator}
\end{figure}

\begin{figure}[h!]
  \centering
    \includegraphics[width=\textwidth]{block_diagrams/win_gen_interior.pdf}
  \caption{Interior Block Diagram of the Window Generator}
\end{figure}

\begin{figure}[h!]
  \centering
    \includegraphics[width=\textwidth]{block_diagrams/acounter.pdf}
  \caption{Block diagram of a counter used in window generation}
\end{figure}

\begin{figure}[h!]
  \centering
    \includegraphics[width=\textwidth]{block_diagrams/bcounter.pdf}
  \caption{Block diagram of a counter used in window generation}
\end{figure}

\begin{figure}[h!]
  \centering
    \includegraphics[width=\textwidth]{state_diagrams/diff_counter.pdf}
  \caption{State Diagram for the diff counter, Moore machine: signals
    \texttt{max=c\_cuis=max\_itr}, \texttt{leap=iter\_count=v\_leap}. Omit
    unused signals (X) for compactness. Colorcoded signal bundles.}
\end{figure}

\begin{figure}[h!]
  \centering
    \includegraphics[width=\textwidth]{state_diagrams/diff_window_gen.pdf}
  \caption{State Diagram for the window generator code, Moore machine:
    signals \texttt{max=c\_cuis=max\_itr},
    \texttt{leap=iter\_count=v\_leap}. Omit unused signals (X) for
    compactness.}
\end{figure}

\begin{figure}[h!]
  \centering
    \includegraphics[width=\textwidth]{timing_diagrams/diff_counter.pdf}
  \caption{Timing diagram of the Difference counter}
\end{figure}



% ----------------------------------------
\subsection{Iterative Function Module (IFM)}

Rendering Julia set fractals requires many iterations of relatively
simple computations in the complex plane. This sequence of
computations is independent for each point in the image, which is why
the calculation of fractal sets lends itself to parallel
computation. However, the very nature of the iterated fractal
calculation means that the run lengths of the individual calculations
are heterogenous, introducing synchronization issues.

With a pair of queues handling the heterogenity of the fractal
calculations, the IFM merely need calculate applications of the function:

\begin{equation}
f(z) = z^2 + c
\end{equation}

repeatedly, where $z,c \in \mathbb{C}$.

The complex nature of the calculations means that there need be 3
multiplications, not just 1, to find the next iteration of $z$.

Numbers are represented as two's-complement fixed-point binary
values. We restrict ourselves to 18 bits, as the onboard multipliers
are sized such. In order to accomodate the largest-magnitude value we
expect to come across during any iteration, we require 6 bits to the
left of the radix. Thus, our fixed-point values ahve 12 bits to the
right of the radix.

We count 127 iterations: after the 127th iteration, we consider the
originating point bounded.

We pass a pixel position and point on the complex plane to the IFM
through the queue. The complex plane position is meant to
facilitate the fractal calculation and the pixel position is meant for
display on the screen.

% ----------------------------------------
% figures for the IFMs

\begin{figure}[h!]
  \centering
    \includegraphics[width=\textwidth]{block_diagrams/ifm.pdf}
  \caption{Block diagram for the IFM's themselves}
\end{figure}

\begin{figure}[h!]
  \centering
    \includegraphics[width=\textwidth]{state_diagrams/ifm.pdf}
  \caption{State diagram for a single IFM unit (IFM itself), Moore
    machine: using abstract transition descriptions. Omits unused
    signals for compactness.}
\end{figure}

\begin{figure}[h!]
  \centering
    \includegraphics[width=\textwidth]{block_diagrams/ifmunit.pdf}
  \caption{Block diagram for the IFM wrappers}
\end{figure}

\begin{figure}[h!]
  \centering
    \includegraphics[width=\textwidth]{state_diagrams/ifmunit.pdf}
  \caption{State diagram for a single IFM unit (IFM handler/wrapper),
    Moore machine: using abstract transition descriptions.}
\end{figure}

\begin{figure}[h!]
  \centering
    \includegraphics[width=\textwidth]{timing_diagrams/gen_ifm.pdf}
  \caption{Timing diagram of interface between the window generator
    and the IFMs.}
\end{figure}



% ----------------------------------------
\subsection{Parallel IFM Control Module}

% !!! NEED SOMEONE TO WRITE SOMETHING ABOUT THIS



% ----------------------------------------
\subsection{Coordinate-Breakaway Lookup Table}

After the count associated with each pixel is calculated, it must be
stored in a framebuffer that interfaces both with the Parallel IFM
Control Module as well as the VGA module.

For this, we use the SRAM chip that is built into the DE2 board, for
its relatively expansive memory size (versus on-chip memory), fast
speed, and ease of use (versus the SDRAM chip). The SRAM chip contains
512kibibytes in total, and can be accessed and written to in half a 50MHz
clock cycle, making it ideal for our purposes.

Since we display a $640\times 480$ image in the VGA module and keep 8
bits of iteration information for each pixel, we need a grand total of
300kibibytes to store the information, fitting well within the
confines of the given 512kibibyte SRAM chip.

We use a straightforward addressing scheme to store the count
information, using the $y$ position as the top 9 bits of the address,
and the $x$ position as the bottom 10 bits of the address. This way,
finding the address from a given pixel position is very fast.

A small wrinkle is the fact the SRAM is in fact a 256Kx16 bit memory,
reading and writing in 16 bit chunks. This merely means that the very
bottom bit of the $x$ position does not go to the address, but is
routed to the bitmask signal which controls which bits should be
exposed or mutated, if either.

\subsection{Reading/Writing}

Since the SRAM has only one multiplexed port, reads and writes must be
multiplexed. Given that the VGA module will be requesting data for
long stretches of time at a frequency around half of 50MHz, this means
that we can interleave reads and writes to the SRAM.

However, we can use the structure of the reads from the VGA to our
advantage, and allow more writes. Reads always follow a pattern, where
if we read the lower half of a 16bit word, then we will read the
higher half in the next 25MHz clock cycle. Hence, if we read the lower
half of a word, we can fetch the entire word in one read, save the
higher half in a register, and return the higher half when asked in
the next clock cycle.

Hence, the SRAM can write every other time it would normally be
writing, meaning we can interleave 3 writes for each read for superior
bandwidth.

This is also why we used the particular addressing scheme we did: if
$y$ took the lower address bits, then we could not rely on the
regularity of the read accesses incrementing with $x$, and we would be
reduced back to interleaving single reads and writes.

However, we still have one out of four 50MHz cycles dedicated to
reading from the SRAM (lest the VGA module not get the required value,
and snow crash the rest of the screen\footnote{One could conceivably
  get away with putting writes aboves reads in precedence, if one
  accepts a glitchy screen displaying for split seconds, with a
  pronounced effect if the parameters change frequently}), which
necesitates a queue to write to the SRAM. We use a shift register,
which only shifts forward when not reading, and always accepts new
values onto the end\footnote{The queue will only block for one cycle
  in between several non-blocked cycles, so this is fine}. This is not
perfect: if the input queue hands off values every clock cycle for an
extended amount of time while the SRAM is contested for reads, then we
will lose values. However, we don't believe this will be too large a
problem.

Hence, the Coordinate-Breakaway Lookup Table abstracts away the SRAM,
serving readers and writers in an abstract manner.

% ----------------------------------------
% figures for the CLUT

\begin{figure}[h!]
  \centering
    \includegraphics[width=\textwidth]{block_diagrams/clut.pdf}
  \caption{Block Diagram of the Coordinate-Breakaway LUT: signals on
    the right side are inputs, signals on the left are outputs.}
\end{figure}

\begin{figure}[h!]
  \centering
    \includegraphics[width=\textwidth]{state_diagrams/clut.pdf}
  \caption{Simplified State Diagram of the Coordinate-Breakaway LUT:
    Mealy machine, only includes re and we as inputs and rv as an
    output, with ~ denoting a lack of outputs}
\end{figure}

\begin{figure}[h!]
  \centering
    \includegraphics[width=\textwidth]{timing_diagrams/ifm_clut.pdf}
  \caption{Timing diagram of the interface between the IFMs and the CLUT.}
\end{figure}




% ----------------------------------------
\subsection{VGA Module}

A VGA controller is connected to the Coordinate-Breakaway lookup table
in order to display the generated Julia set. As the controller cycles
through output coordinates within the display area, it modifies the
read address signal for the lookup table. The data signal coming from
the RAM is thus the breakaway value associated with that coordinate.

This breakaway value is passed through a decoder known as the
Colorization Lookup Table and the resulting $(R, G, B)$ signal tuple
is sent to the VGA.

% ----------------------------------------
% figures for the VGA

\begin{figure}[h!]
  \centering
    \includegraphics[width=\textwidth]{block_diagrams/vga.pdf}
  \caption{Block diagram of the VGA module}
\end{figure}

\begin{figure}[h!]
  \centering
    \includegraphics[width=\textwidth]{state_diagrams/vga.pdf}
  \caption{State diagram for the \texttt{VGA} module, Mealy machine: ~
    stands in for no input/output, and VGA stands in for all the VGA\_
    signals (\texttt!VGA\_CLK!, \texttt!VGA\_HS!, \texttt!VGA\_VS!,
    \texttt!VGA\_BLANK!, \texttt!VGA\_SYNC!, \texttt!VGA\_R!, \texttt!VGA\_G!,
    \texttt!VGA\_B!). Omits unused signals for compactness.}
\end{figure}

\begin{figure}[h!]
  \centering
    \includegraphics[width=\textwidth]{timing_diagrams/clut_vga.pdf}
  \caption{Timing diagram of the interface between the CLUT and VGA module}
\end{figure}



% ------------------------------------------------------------
\section{Parametrization Modules}

These modules can be used parameterize the output fractal

\begin{itemize}
\item PS/2 Keyboard input can be used to allow the user to specify
  fractal recomputation using a different set of parameters,
  permitting the modification of window ranges and Julia Set
  constants. Keyboard input would be facilitated through the Nios II
  processor and would require a reexecution of the Window Generator.
\item We could create a module for permuting the display colors given
  by the Colorization Lookup table using a periodic function, thus
  causing the colors to cycle. This module would allow us to modify
  the way the fractal looks without recomputing it.
\item Spectral analysis module takes audio input and uses it to
  influence the behavior of the Color Permutation module.
\end{itemize}


% ------------------------------------------------------------
% !!! UPDATE MILESTONES IF NECESSARY
\section{Updated Milestones}

\begin{itemize}
\item Milestone 1 (Mar 27): \\
  Develop a Window Generator and
  Parallelized IFM module that can communicate successfully.
\item Milestone 2 (Apr 10): \\
  Display the colorized (and static) Julia set through VGA.
\item Milestone 3 (Apr 24): \\ 
  Implement parameter mutation, with subsequent updates to the
  displayed Julia set.
\item Final Report and Presentation (May 10)
\end{itemize}

\end{document}
